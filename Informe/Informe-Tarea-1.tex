\documentclass[letterpaper,10pt]{article}

\usepackage[utf8]{inputenc}
\usepackage[spanish]{babel}
\usepackage{fontenc}
\usepackage[dvipdfmx]{graphicx}
\usepackage{bmpsize,wrapfig,xcolor}
\usepackage{fullpage}

% Para evitar que se indente solo a cada rato
\setlength\parindent{0pt}

\begin{document}
	\begin{titlepage}

		\begin{wrapfigure}{R}{0.3\textwidth}
			\includegraphics[width=0.3\textwidth]{logoFCFM.png}
		\end{wrapfigure}

		\noindent \phantom - % "Hax" para que quede alineada la imagen con el texto

		Universidad de Chile

		Facultad de Ciencias Físicas y Matemáticas

		Depto. de Ciencias de la Computación

		CC4102 - Diseño y Análisis de Algoritmos

		\vfill

		\begin{center}
			\begin{Huge}
				{\textbf{Tarea 1}}
			\end{Huge}
		\end{center}

		\vfill

		\begin{flushright}
			\begin{tabular}{lll}
				Integrantes	&:	& Rodrigo Delgado\\
						&	& Belisario Panay\\
						&	& Gabriel Sanhueza\\
				Profesor	&:	& Gonzalo Navarro\\
				Ayudante	&:	& Sebastián Ferrada\\
				Auxiliar	&:	& Jorge Bahamondes\\
			\end{tabular}
		\end{flushright}

	\end{titlepage}

	% % % % % % % % % % % % % % % % % % % % % % % % % % % % % % % % % % % % % % % % % % % % % % % % % % % % % % % % % % % % % % % % % % % % % % % % % % % % % % % % % % % % % % % % % %
	\pagebreak
	% % % % % % % % % % % % % % % % % % % % % % % % % % % % % % % % % % % % % % % % % % % % % % % % % % % % % % % % % % % % % % % % % % % % % % % % % % % % % % % % % % % % % % % % % %

	\section*{Hipótesis}
% 	TODO Hipótesis

	Aquí deberían ir:

	\begin{itemize}
		\item Especificaciones de la máquina a usar (cof, cof, anakena, cof, cof)
		\item Sistema Operativo (Ubuntu 14.04 ??)
		\item Lenguaje usado (Java ??)
		\item Compilador (JVM ??)
		\item RAM (Caleta xDDD)
		\item Características del disco duro (ahí no sé, probablemente sea un disco estándar de 5400 RPM)
	\end{itemize}

	Usamos M = 4096 KB, tamaño de una página de disco (no sé si es 4096, pero es bastante estándar para Linux)
	m = 40\% de M para el split

	% % % % % % % % % % % % % % % % % % % % % % % % % % % % % % % % % % % % % % % % % % % % % % % % % % % % % % % % % % % % % % % % % % % % % % % % % % % % % % % % % % % % % % % % % %
	\pagebreak
	% % % % % % % % % % % % % % % % % % % % % % % % % % % % % % % % % % % % % % % % % % % % % % % % % % % % % % % % % % % % % % % % % % % % % % % % % % % % % % % % % % % % % % % % % %

	\section*{Diseño Experimental}
% 	TODO Diseño Experimental

	Explicar cómo diseñamos el R-Tree y que debería hacer cada método/función

	% % % % % % % % % % % % % % % % % % % % % % % % % % % % % % % % % % % % % % % % % % % % % % % % % % % % % % % % % % % % % % % % % % % % % % % % % % % % % % % % % % % % % % % % % %
	\pagebreak
	% % % % % % % % % % % % % % % % % % % % % % % % % % % % % % % % % % % % % % % % % % % % % % % % % % % % % % % % % % % % % % % % % % % % % % % % % % % % % % % % % % % % % % % % % %

	\section*{Presentación de los Resultados}
%	TODO Presentación de los Resultados

	\subsection*{Tiempo de Construcción del R-Tree}
	\subsubsection*{Variante 1 de Insertar: (Inserte nombre aquí)}
% 	TODO Gráficos?
	Hacer un gráfico?

	\subsubsection*{Variante 2 de Insertar: (Inserte nombre aquí)}
% 	TODO Gráficos?
	Hacer otro gráfico?

	\subsection*{Espacio ocupado y porcentaje de llenado de páginas de disco}
% 	TODO Gráficos?
	Aún más gráficos?

	\subsection*{Desempeño de operación \textit{Buscar}}

	% % % % % % % % % % % % % % % % % % % % % % % % % % % % % % % % % % % % % % % % % % % % % % % % % % % % % % % % % % % % % % % % % % % % % % % % % % % % % % % % % % % % % % % % % %
	\pagebreak
	% % % % % % % % % % % % % % % % % % % % % % % % % % % % % % % % % % % % % % % % % % % % % % % % % % % % % % % % % % % % % % % % % % % % % % % % % % % % % % % % % % % % % % % % % %

	\section*{Análisis e Interpretación}
% 	TODO Análisis e Interpretación
	Aquí hay que hacer el chamullo correspondiente (?)

\end{document}
