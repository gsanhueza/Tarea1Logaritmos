\documentclass[letterpaper,10pt]{article}

\usepackage[utf8]{inputenc}
\usepackage[spanish]{babel}
\usepackage{fontenc}
\usepackage[dvipdfmx]{graphicx}
\usepackage{bmpsize,wrapfig,xcolor}
\usepackage{fullpage}
\usepackage{amssymb}

% Para evitar que se indente solo a cada rato
\setlength\parindent{0pt}

\begin{document}
	\begin{titlepage}

		\begin{wrapfigure}{R}{0.3\textwidth}
			\includegraphics[width=0.3\textwidth]{logoFCFM.png}
		\end{wrapfigure}

		\noindent \phantom - % "Hax" para que quede alineada la imagen con el texto

		Universidad de Chile

		Facultad de Ciencias Físicas y Matemáticas

		Depto. de Ciencias de la Computación

		CC4102 - Diseño y Análisis de Algoritmos

		\vfill

		\begin{center}
			\begin{Huge}
				{\textbf{Tarea 1}}
			\end{Huge}
		\end{center}

		\vfill

		\begin{flushright}
			\begin{tabular}{lll}
				Integrantes	&:	& Rodrigo Delgado\\
				&	& Belisario Panay\\
				&	& Gabriel Sanhueza\\
				Profesor	&:	& Gonzalo Navarro\\
				Ayudante	&:	& Sebastián Ferrada\\
				Auxiliar	&:	& Jorge Bahamondes\\
			\end{tabular}
		\end{flushright}

	\end{titlepage}

	% % % % % % % % % % % % % % % % % % % % % % % % % % % % % % % % % % % % % % % % % % % % % % % % % % % % % % % % % % % % % % % % % % % % % % % % % % % % % % % % % % % % % % % % % %
	\pagebreak
	% % % % % % % % % % % % % % % % % % % % % % % % % % % % % % % % % % % % % % % % % % % % % % % % % % % % % % % % % % % % % % % % % % % % % % % % % % % % % % % % % % % % % % % % % %

	\section*{Hipótesis}
	% 	TODO Hipótesis

	\subsection*{Especificaciones de la máquina utilizada (Anakena)}

	\begin{itemize}
		\item Procesador: Intel \textregistered Xeon \textregistered CPU E5620 @ 2.40GHz
		\item Arquitectura: x86\_64
		\item Número de CPUs: 12
		\item Memoria RAM: 12288 KB
		\item Tamaño de página de disco: M = 512 bytes ($\ulcorner40\% M\urcorner = 205$)
		\item Sistema Operativo: Linux version 4.4.6-gentoo
		\item Lenguaje usado: C
		\item Compilador: gcc version 4.9.3
	\end{itemize}


	% 	FIXME
	% 	que limitamos el tamaño de los nodos a 128 rectangulos y que a lo mas va a tener 127, que nuca tenemos mas de dos archivos abiertos
	% 	y que el random lo utilizamos siempre con la misma semilla para tener experimentos repetibles

	% % % % % % % % % % % % % % % % % % % % % % % % % % % % % % % % % % % % % % % % % % % % % % % % % % % % % % % % % % % % % % % % % % % % % % % % % % % % % % % % % % % % % % % % % %
	\pagebreak
	% % % % % % % % % % % % % % % % % % % % % % % % % % % % % % % % % % % % % % % % % % % % % % % % % % % % % % % % % % % % % % % % % % % % % % % % % % % % % % % % % % % % % % % % % %

	\section*{Diseño Experimental}
	% 	TODO Diseño Experimental

	Explicar cómo diseñamos el R-Tree y que debería hacer cada método/función

	% % % % % % % % % % % % % % % % % % % % % % % % % % % % % % % % % % % % % % % % % % % % % % % % % % % % % % % % % % % % % % % % % % % % % % % % % % % % % % % % % % % % % % % % % %
	\pagebreak
	% % % % % % % % % % % % % % % % % % % % % % % % % % % % % % % % % % % % % % % % % % % % % % % % % % % % % % % % % % % % % % % % % % % % % % % % % % % % % % % % % % % % % % % % % %

	\section*{Presentación de los Resultados}
	%	TODO Presentación de los Resultados

	\subsection*{Tiempo de Construcción del R-Tree}
	\subsubsection*{Variante 1 de Insertar: (Inserte nombre aquí)}
	% 	TODO Gráficos?
	Hacer un gráfico?

	\subsubsection*{Variante 2 de Insertar: (Inserte nombre aquí)}
	% 	TODO Gráficos?
	Hacer otro gráfico?

	\subsection*{Espacio ocupado y porcentaje de llenado de páginas de disco}
	% 	TODO Gráficos?
	Aún más gráficos?

	\subsection*{Desempeño de operación \textit{Buscar}}

	% % % % % % % % % % % % % % % % % % % % % % % % % % % % % % % % % % % % % % % % % % % % % % % % % % % % % % % % % % % % % % % % % % % % % % % % % % % % % % % % % % % % % % % % % %
	\pagebreak
	% % % % % % % % % % % % % % % % % % % % % % % % % % % % % % % % % % % % % % % % % % % % % % % % % % % % % % % % % % % % % % % % % % % % % % % % % % % % % % % % % % % % % % % % % %

	\section*{Análisis e Interpretación}
	% 	TODO Análisis e Interpretación
	Aquí hay que hacer el chamullo correspondiente (?)

\end{document}
