\documentclass[letterpaper,10pt]{article}

\usepackage[utf8]{inputenc}
\usepackage[spanish]{babel}
\usepackage{fontenc}
\usepackage[dvipdfmx]{graphicx}
\usepackage{bmpsize,wrapfig,xcolor}
\usepackage{fullpage}
\usepackage{amssymb}
\usepackage[hidelinks]{hyperref}

% Para evitar que se indente solo a cada rato
\setlength\parindent{0pt}

\begin{document}
	\begin{titlepage}

		\begin{wrapfigure}{R}{0.3\textwidth}
			\includegraphics[width=0.3\textwidth]{logoFCFM.png}
		\end{wrapfigure}

		\noindent \phantom - % "Hax" para que quede alineada la imagen con el texto

		Universidad de Chile

		Facultad de Ciencias Físicas y Matemáticas

		Depto. de Ciencias de la Computación

		CC4102 - Diseño y Análisis de Algoritmos

		\vfill

		\begin{center}
			\begin{Huge}
				{\textbf{Tarea 1}}
			\end{Huge}
		\end{center}

		\vfill

		\begin{flushright}
			\begin{tabular}{lll}
				Integrantes	&:	& Rodrigo Delgado\\
						&	& Belisario Panay\\
						&	& Gabriel Sanhueza\\
				Profesor	&:	& Gonzalo Navarro\\
				Ayudante	&:	& Sebastián Ferrada\\
				Auxiliar	&:	& Jorge Bahamondes\\
			\end{tabular}
		\end{flushright}

	\end{titlepage}

	% % % % % % % % % % % % % % % % % % % % % % % % % % % % % % % % % % % % % % % % % % % % % % % % % % % % % % % % % % % % % % % % % % % % % % % % % % % % % % % % % % % % % % % % % %
	\newpage
	% % % % % % % % % % % % % % % % % % % % % % % % % % % % % % % % % % % % % % % % % % % % % % % % % % % % % % % % % % % % % % % % % % % % % % % % % % % % % % % % % % % % % % % % % %

	\tableofcontents

	% % % % % % % % % % % % % % % % % % % % % % % % % % % % % % % % % % % % % % % % % % % % % % % % % % % % % % % % % % % % % % % % % % % % % % % % % % % % % % % % % % % % % % % % % %
	\newpage
	% % % % % % % % % % % % % % % % % % % % % % % % % % % % % % % % % % % % % % % % % % % % % % % % % % % % % % % % % % % % % % % % % % % % % % % % % % % % % % % % % % % % % % % % % %

	\section{Hipótesis}

	\subsection*{Especificaciones de la máquina utilizada (Anakena)}

	\begin{itemize}
		\item Procesador: Intel \textregistered Xeon \textregistered CPU E5620 @ 2.40GHz
		\item Arquitectura: x86\_64
		\item Número de CPUs: 12
		\item Memoria RAM: 12288 KB
		\item Tamaño de página de disco: M = 512 bytes ($\ulcorner40\% M\urcorner = 205$)
		\item Sistema Operativo: Linux version 4.4.6-gentoo
		\item Lenguaje usado: C
		\item Compilador: gcc version 4.9.3
	\end{itemize}

	Usaremos siempre la misma semilla de aleatoriedad, para poder tener experimentos ``aleatorios'' repetibles.

	Limitamos la cantidad máxima de rectángulos en un nodo con respecto al tamaño de página del disco.

	Así, si \textit{BLOCK\_SIZE} = 4096 y tenemos un \textit{struct rectangle} llamado \textit{Rectangle}:

% 	FIXME Decidir si usar 512 o 4096
	\begin{itemize}
		\item M = \textit{BLOCK\_SIZE / sizeof(struct rectangle)} = $ 4096 / 32 = 128 $
		\item m = 40\% de M. = $ 40\% * 128 = 51 $
	\end{itemize}

	Por último, la idea es nunca tener más de dos archivos abiertos en un instante dado.

	% % % % % % % % % % % % % % % % % % % % % % % % % % % % % % % % % % % % % % % % % % % % % % % % % % % % % % % % % % % % % % % % % % % % % % % % % % % % % % % % % % % % % % % % % %
	\newpage
	% % % % % % % % % % % % % % % % % % % % % % % % % % % % % % % % % % % % % % % % % % % % % % % % % % % % % % % % % % % % % % % % % % % % % % % % % % % % % % % % % % % % % % % % % %

	\section{Diseño Experimental}

	En nuestra implementación hicimos 2 \textit{structs} para manejar los rectángulos en el R-Tree.

	\subsection{Structs}

	\subsubsection{Rectangle}

	Esta estructura posee información sobre:
	\begin{itemize}
		\item Coordenada X.
		\item Coordenada Y.
		\item Ancho del rectángulo.
		\item Alto del rectángulo.
		\item Identificador (nombre) del rectángulo.
		\item Identificador del hijo de este rectángulo.
	\end{itemize}

	\subsubsection{Node}

	Esta estructura posee información sobre:
	\begin{itemize}
		\item Arreglo dinámico de rectángulos.
		\item Tamaño del arreglo.
		\item Nombre del nodo actual (para uso como nombre de archivo en disco).
	\end{itemize}

	\subsection{Constantes}

	\begin{itemize}
		\item BLOCK\_SIZE: Tamaño del bloque en disco.
		\item count: Variable estática para diferenciar nodos al escribirlos a disco.
		\item M: BLOCK\_SIZE / sizeof(Rectangle) --- Máximo número de rectángulos en un nodo.
		\item m: 40\% de M --- Mínimo número de rectángulos en un nodo.
	\end{itemize}

	\subsection{Funciones}

	\begin{itemize}
		\item \textit{Rectangle* createRectangle(int x, int y, int w, int h, int id)}: Crea un rectángulo.
		\item \textit{Node* createNode()}: Crea un nodo.
		\item \textit{Node* loadFromDisk(char *filename)}: Carga el archivo \textit{filename} desde el disco.
		\item \textit{char* writeToDisk(Node *data)}: Escribe la información del nodo al disco.
		\item \textit{int intersect (Rectangle *r1, Rectangle *r2)}: Revisa si dos rectángulos se intersectan.
		\item \textit{void mergeRectangle(Rectangle *r1, Rectangle *r2)}: Actualiza las coordenadas de r1 al agregarle r2.
		\item \textit{int MBR(Rectangle *r1, Rectangle *r2)}: Calcula la nueva Area si se agrega r2 a r1.
		\item \textit{void printRectangle(Rectangle *r)}: Imprime información de un rectángulo.
		\item \textit{void insert(char *nodeName , Rectangle *r)}: Inserta un rectángulo al nodo dado.
		\item \textit{Node *search(char *nodeName, Rectangle *rect)}: Busca en el nodo todos los rectángulos que intersectan a *rect.
		\item \textit{Rectangle ** linearSplit(Node *header)}: Control de overflow usando Linear Split.
		\item \textit{Rectangle ** greeneSplit(Node *header)}: Control de overflow usando Greene Split.
		\item Rectangle ** controlOverFlow(Node *header, Rectangle *r); \textbf{TODO}
		\item int *calculateBounds(Node *pNode); \textbf{TODO}
		\item Rectangle **calculateXRectangles(Node *pNode); \textbf{TODO}
		\item Rectangle **makeRandom(Node pNode); \textbf{TODO}
	\end{itemize}

	% % % % % % % % % % % % % % % % % % % % % % % % % % % % % % % % % % % % % % % % % % % % % % % % % % % % % % % % % % % % % % % % % % % % % % % % % % % % % % % % % % % % % % % % % %
	\newpage
	% % % % % % % % % % % % % % % % % % % % % % % % % % % % % % % % % % % % % % % % % % % % % % % % % % % % % % % % % % % % % % % % % % % % % % % % % % % % % % % % % % % % % % % % % %

	\section{Presentación de los Resultados}
	%	TODO Presentación de los Resultados

	\subsection{Tiempo de Construcción del R-Tree}
	\subsubsection{Linear Split}
	% 	TODO Gráficos?
	Hacer un gráfico?

	\subsubsection{Greene Split}
	% 	TODO Gráficos?
	Hacer otro gráfico?

	\subsection{Espacio ocupado y porcentaje de llenado de páginas de disco}
	% 	TODO Gráficos?
	Aún más gráficos?

	\subsection{Desempeño de operación \textit{Buscar}}

	% % % % % % % % % % % % % % % % % % % % % % % % % % % % % % % % % % % % % % % % % % % % % % % % % % % % % % % % % % % % % % % % % % % % % % % % % % % % % % % % % % % % % % % % % %
	\newpage
	% % % % % % % % % % % % % % % % % % % % % % % % % % % % % % % % % % % % % % % % % % % % % % % % % % % % % % % % % % % % % % % % % % % % % % % % % % % % % % % % % % % % % % % % % %

	\section{Análisis e Interpretación}
	% 	TODO Análisis e Interpretación
	Aquí hay que hacer el chamullo correspondiente (?)

\end{document}
